\section{Постановка задачи}
Целью данной работы является исследование существующих методов извлечения структуры и полемической позиции аргументации и разработка новых модельных подходов. 

Перед дальнейшими определениями необходимо пояснить термен "компоненты аргументации". Определение и набор компонент разнятся от работы к работе, но в общем случае можно выделить следующие сущности:

\begin{itemize}
    \item Фокус - центральный объект обсуждения.
    \item Тема - противоречивое утверждение относительно фокуса.
    \item Утверждение - фраза или предложение, поддерживающие или опровергающее тему.
    \item Предпосылка - фраза или предложение, поддерживающие или опровергающие или тему или предложение, основанное на фактах, а не на убеждении.
\end{itemize}

Под структурой аргумента подразумевается набор компонент аргументации и отношения связности или релевантности между ними. Например для темы "запрет ядерной энергии" утверждение "ядерная энергия вредна для окружающей среды" является релевантным, а утверждение "человеческий глаз воспринимает около миллиона оттенков" релевантным не является. Отношение релевантности является бинарным отношением между компонентами аргументации.

Задача определения полемической позиции заключается в определении типа связи между релевантными компонентами аргументации. Релевантное теме утверждение может как соглашаться с темой, так и ее опровергать. Задача классификации связей между релевантными компонентами аргументации в два класса поддержки и атаки называется определением полемической позиции. Как видно из определения, полемическая позиция применима исключительно к релевантным компонентам аргументации и представляет собой два взаимоисключающих класса связей.

Для достижения поставленной в работе цели необходимо решить следующие подзадачи:

\begin{enumerate}
	\item Произвести обзор существующих решений и корпусов, выделить наиболее подходящие постановки.
    \item Воспроизвести избранные работы или получить новые базовые решения.
    \item Предложить или адаптировать новые модельные подходы для извлечения аргументации.
    \item Провести анализ полученной модели с целью интерпретации полученных результатов.
\end{enumerate}
