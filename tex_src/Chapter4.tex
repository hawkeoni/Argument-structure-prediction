\section{Заключение}
В рамках данной работы были получены следующие результаты:
\begin{itemize}
    \item Проведен обзор существующих методов извлечения аргументации.
    \item Обучены системы, основанные на больших предобученных языковых моделях, для извлечения аргументации.
    \item Реализован базовый вариант переноса знаний на русский язык.
    \item Реализовано базовое веб-приложение для извлечения аргументации.
\end{itemize}

Несмотря на то, что для корпусов специфичных для задачи аргументации отсутствуют массовые замеры качества различных архитектур, какие часто имеют устоявшиеся в научной сфере корпуса (wikitext8, CONLL2003, WSJ PennTreebank), обученные модели показывают более высокое качество, чем модели обученных в исходных работах. Данный факт во многом объясняется тем, что модели, предоставленные в данной работе, базируются на больших предобученных трансформерах, которые вышли позже публикаций с корпусами. 

Рассматриваемые в данной работе базовые архитектуры имеют внутри себя определенный набор знаний о мире и сильную обобщающую способность, полученную за счет обучения на сверхбольших корпусах. Подобное предобучение является одной из главных причин, по которым системы предложенные в рамках этой работы работают лучше небольших нейросетевых моделей, обученных в рамках исходного корпуса, или методов машинного обучения.


\subsection{Планы по дальнейшей работе}
В качестве дальнейших вариантов развития работы можно предложить следующие направления:
\begin{itemize}
    \item Применение обученных моделей с целью исследования их эффективности на соревновательных корпусах CLEF, FEVER и их аналогах.
    \item Исследование улучшения качества работы моделей за счет адаптации новых корпусов и радикального увеличения базы размеченных коллекций, в том числе за счет задач NLI.
    \item Исследование подхода улучшения качества работы моделей за счет мультизадачного обучения структуре оформления аргументации, знаниям о мире и совпадении оценки тональности в утверждении и обосновании.
    \item Применение более эффективных и новых мультиязычных языковых моделей в качестве базовой архитектуры с целью их сравнительного анализа.
\end{itemize}
