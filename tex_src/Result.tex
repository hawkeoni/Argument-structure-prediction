\section{Заключение}

В рамках данной работы были получены следующие результаты:
\begin{itemize}
    \item Проведен обзор существующих методов и корпусов для извлечения аргументации.
    \item Обучены системы, основанные на больших предобученных языковых моделях, для извлечения аргументации.
    \item Реализован базовый вариант переноса знаний на русский язык для задачи извлечения структуры аргументации.
    \item Исследована возможность использования предобучения на задаче логического вывода для улучшения результатов на задачах извлечения аргументации.
    \item Реализована система для обнаружения пропаганды в новостных ресурсах.
    \item Реализовано базовое веб-приложение для взаимодействия с моделями.
\end{itemize}

В данной работе исследованы задачи автоматического выявления аргументации на примере извлечения структуры и определении полемической позиции аргументации. Несмотря на отсутствие крупных корпусов, на которых исследовательские группы целеноправленно сравнивали бы свои модели, были выделены избранные коллекции. На коллекциях ArgsEN \cite{toledo2020multilingual} в задаче определения полемической позиции аргументации был улучшен результат англоязычной модели с 89.3 пунктов макро F1-меры до 90.3; на коллекции IBM Evidence Search \cite{shnarch2018will} для определения структуры аргументации также улучшен результат с 76 пунктов микро точности до 81.1. На коллекциях, где отсутствовали авторские замеры были также предоставлены показатели качества: на корпусе EviEN на задаче определения полемической позиции - 67.6, на задаче определения структуры аргумента (релевантности) - 72.8.

На корпусе IBM Evidence Search было проведено исследование возможности использования мультиязычной модели для переноса знаний на русский язык. Для этого тестовая выборка была переведена на русский язык. Модель показывает падение точности с 80.5 пунктов для англоязычного корпуса до 76.4 пунктов, что отражает ожидаемое ухудшение качество при смене языка. Тем не менее качество этой модели достигает современного уровня извлечения аргументации. 

Дополнительно на корпусе IBM Evidence Search была исследована возможность применения закономерностей, выученных моделью на корпусе логического вывода. Использование корпуса TERRa в сценарии без дообучения показывает возможность применения подобного решения на задаче определения структуры аргументации. Обучение одновременно логическому выводу и структуре аргументации приводит к ухудшению качества, полученного исключительно на корпусе аргументации, что говорит о непригодности использования корпуса TERRa в подобном сценарии. Аналогичные испытания проводились и для корпуса ArgsEN на задаче определения полемической позиции и корпусе SNLI для логического вывода. В этом эксперименте дообученная модель также привела к ухудшению качества по сравнению с моделью, использующей исключительно корпус аргументации.

Дополнительно исследована задача извлечения пропаганды - частного вида аргументации. Предоставлены модели, занявшие 6 и 7 место в научном семинаре Semeval 11 \cite{dimov-etal-2020-nopropaganda}.

Для взаимодействия с моделями предоставлен веб-сервис, позволяющий для пар компонент аргументации оценить релевантность и полемическую позицию.

В качестве будущего развития работы можно предложить:
\begin{itemize}
    \item Дополнение моделей внешними знаниями для моделирования более сложных зависимостей.
    \item Использование полученных аргументов для создания карт аргументации или генерации текста.
    \item Создание русскоязычной коллекции для решения задачи извлечения аргументации.
\end{itemize}